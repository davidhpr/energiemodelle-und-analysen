\documentclass{eegreport}

%% my usepackages
\usepackage{ltablex}
\usepackage[official]{eurosym}
\usepackage{listings}
\usepackage{amsmath, amsthm}
\usepackage{amsmath}
%\usepackage{makecell}
%\usepackage{pgfplots} gemeinsam mit matlab2tikz verwenden
	
% external files and paths
\addbibresource{Literature.bib} %% remove, if using BibTeX instead of biblatex
\graphicspath{{graphics/}}

%% general metadata:
\newcommand{\mytitle}{VU Energiemodelle und Analysen (373.011)}  %% also used for PDF metadata (hyperref)
\newcommand{\mysubject}{Protokoll Übung 2}  %% also used for PDF metadata (hyperref)
\newcommand{\myauthors}{Ivan Grubesic (01425089), Michael Kern (0935115), David Pribic (01428675), Ermin Sefer (01525021)}  %% also used for PDF metadata (hyperref)
\newcommand{\mysupervisor}{Ao.Univ.Prof. Univ.Prof. Dipl.-Ing. Dr.techn. Haas, Privatdoz. Dipl.-Ing. Dr.techn. Auer, Univ.Ass. Dipl.-Ing. Perger}  %% also used for PDF metadata (hyperref)
\newcommand{\mydate}{April 2020}  %% also used for PDF metadata (hyperref)

\newcommand{\mykeywords}{KEYWORDS}  %% also used for PDF metadata (hyperref)

%% this information is used only for generating the title page:
\newcommand{\myuniversity}{TU Wien} %% your university/school
\newcommand{\myinstitute}{Institut für Energiesysteme und elektrische Antriebe} %% affiliation
\newcommand{\myworkinggroup}{Energy Economics Group} %% working group
\newcommand{\mytown}{Wien} %% your home town
\newcommand{\mymonth}{Mai} %% month you are handing in
\newcommand{\myyear}{2020} %% year you are handing in

%% additional information for generic_documentation title page
\allowdisplaybreaks

\renewcommand{\thesubsubsection}{\alph{subsubsection}}
\begin{document}

\setcounter{section}{2}

\mytitlepage
\pagenumbering{Roman} 
\tableofcontents 
\pagenumbering{arabic} 

\newpage

\subsection{Gewinnmaximierung einer Raffinerie}
Eine Raffinerie stellt 2 Produkte her: Benzin und Diesel. Dieser Raffinerie stehen folgende Ressourcen zur Verfügung:

\begin{enumerate}
\item Arbeitsdauer der Maschinen: max. 1200h
\item Menge an Rohöl: max. 3000 Mengeneinheiten (ME)
\item Arbeitsdauer der Arbeiter: 125h
\end{enumerate}

Zur Herstellung der jeweiligen Produkte sind folgende Ressource erforderlich:

\begin{table}[h]
\begin{center}
\begin{tabular}{|c|c|c|}
\hline 
 & Benzin & Diesel \\ 
\hline 
Arbeitsdauer Maschine & 3h & 2h \\ 
\hline 
Menge an Rohöl & 5 ME & 10 ME \\ 
\hline 
Arbeitsdauer der Arbeiter & 0h & 0.5h \\ 
\hline 
\end{tabular} 
\end{center}
\caption{Ressourcen je Produkt pro Mengeneinheit (ME)}
\label{Ressourcen_Produkt}
\end{table}

Die damit erzielbaren Gewinne in Geldeinheiten (GE) sind:

\begin{table}[h]
\begin{center}
\begin{tabular}{|c|c|c|}
\hline 
 & Benzin & Diesel \\ 
\hline 
Gewinn & 3 GE & 2 GE \\ 
\hline 
\end{tabular} 
\end{center}
\caption{Gewinn je Produkt pro Mengeneinheit (ME)}
\label{Gewinn_Produkt}
\end{table}

\subsubsection{Optimierungsmodell}

Als ersten Schritt stellen wir die Zielfunktion auf. Das ist jene Größe, in unserem Fall der Gewinn, den es hier zu maximieren gilt. Hierbei steht $z$ für den erzielten Gewinn, $x_1$ für die Menge an hergestellten Benzin und $x_2$ für die Menge an hergestellten Diesel.

\begin{equation}
\max_{x_1, x_2}\ z = 3 \cdot x_1 + 4 \cdot x_2
\end{equation}

Die Nebenbedingungen für das Optimierungsmodell lassen sich aus den zur Verfügung stehenden Ressourcen (Maschine, Rohöl, Arbeiter) der Raffinerie und dem Ressourcen je Produkt \ref{Ressourcen_Produkt} aufstellen.

\begin{equation}
3 \cdot x_1 + 2 \cdot x_2 \leq 1200
\end{equation}

\begin{equation}
5 \cdot x_1 + 10 \cdot x_2 \leq 3000
\end{equation}

\begin{equation}
0 \cdot x_1 + 0.5 \cdot x_2 = 125
\label{Nebenbedingung_Arbeiter}
\end{equation}

Nun ergibt sich gleich aus Gleichung \ref{Nebenbedingung_Arbeiter} sehr einfach die zu herstellende Menge an Diesel:

\begin{equation}
x_2 = 250
\end{equation}

Daraus ergibt sich für die beiden anderen Nebenbedingungen:

\begin{equation}
3 \cdot x_1 + 500 \leq 1200
\end{equation}

\begin{equation}
5 \cdot x_1 + 2500 \leq 3000
\end{equation}

Woraus sich folgende Bedingungen für die zu herstellende Menge an Benzin ergeben:

\begin{equation}
x_1 \leq 233,33
\label{Nebenbedingung_Benzin_1}
\end{equation}

\begin{equation}
x_1 \leq 100
\label{Nebenbedinung_Benzin_2}
\end{equation}

Da Gleichung \ref{Nebenbedinung_Benzin_2} eine stärkere Einschränkung ist und dadurch Gleichung  \ref{Nebenbedingung_Benzin_1} bereits erfüllt wird gilt letzendlich die Bedingung \ref{Nebenbedinung_Benzin_2}.\\

Daraus ergibt sich ein maximal erzielbarer Gewinn in Geldeinheiten von:

\begin{equation}
z = 3 \cdot x_1 + 4 \cdot x_2 = 3 \cdot 100 + 4 \cdot 250 = 1300 
\end{equation}

\newpage
\subsubsection{Lösung mittels Simplex-Algorithmus}

Um das Optimierunsproblem mit Hilfe des Simplex-Algorithmus lösen müssen die Simplex-Tableaus angeschrieben werden. Das sieht wie folgt aus:

\myfig{Simplex_1.png}
       {width=1.0\textwidth}
       {Simplex-Tableau 1}
       {Simplex-Tableau 1}
       {Simplex-Tableau 1}
       {h!}
       
Anschließend suchen wir in der letzten Zeile (cT) den größten Wert. Jene Spalte des Wertes ist die sogennante Pivot-Spalte (grün markiert).

\myfig{Simplex_2.png}
       {width=1.0\textwidth}
       {Simplex-Tableau 2}
       {Simplex-Tableau 2}
       {Simplex-Tableau 2}
       {h!}       
       
Anschließend suchen wir die Pivot-Zeile. Das ist jene Zeile mit dem niedrigsten Quotienten aus dem Wert b und dem dazugehörigen Wert aus der Pivot-Spalte. In diesem Fall ist die Zeile 3 die Pivot-Zeile (grün markiert). Das gemeinsame Element von Pivot-Spalte und Pivot-Zeile ist das sogennante Pivot-Element (rot markiert).

\myfig{Simplex_3.png}
       {width=1.0\textwidth}
       {Simplex-Tableau 3}
       {Simplex-Tableau 3}
       {Simplex-Tableau 3}
       {h!}      
       
Nun wird die Pivot-Zeile normiert. Dazu dividiert man diese Zeile durch das Pivot-Element. In diesem Fall dividiert man durch $\dfrac{1}{2}$.

\newpage

\myfig{Simplex_4.png}
       {width=1.0\textwidth}
       {Simplex-Tableau 4}
       {Simplex-Tableau 4}
       {Simplex-Tableau 4}
       {h!}    
       
Anschließend gilt es die anderen Einträge der Pivot-Spalte auf 0 zu transformieren. Dies gelingt uns indem wir die Pivot-Zeile um ein Viefaches von der anderen Zeile abziehen.

\myfig{Simplex_5.png}
       {width=1.0\textwidth}
       {Simplex-Tableau 5}
       {Simplex-Tableau 5}
       {Simplex-Tableau 5}
       {h!}     
       
Da noch nicht alle Koeffizienten der Zielfunktion Null sind (die ersten beiden Koeffizienten von cT) muss dieser Vorgang wiederholt werden. Im folgenden sind nur die Schritte ohne textueller Beschreibung:

\myfig{Simplex_6.png}
       {width=1.0\textwidth}
       {Simplex-Tableau 6}
       {Simplex-Tableau 6}
       {Simplex-Tableau 6}
       {h!} 
       
\textbf{Hinweis:} Falls bei der Bestimmung der Pivot-Zeile ein Koeffizient der Pivot-Spalte 0 ist wird dieser bei der Division nicht berücksichtig!       
    
\myfig{Simplex_7.png}
       {width=1.0\textwidth}
       {Simplex-Tableau 7}
       {Simplex-Tableau 7}
       {Simplex-Tableau 7}
       {h!}        
 
 \myfig{Simplex_8.png}
       {width=1.0\textwidth}
       {Simplex-Tableau 8}
       {Simplex-Tableau 8}
       {Simplex-Tableau 8}
       {h!}             

\myfig{Simplex_9.png}
       {width=1.0\textwidth}
       {Simplex-Tableau 9}
       {Simplex-Tableau 9}
       {Simplex-Tableau 9}
       {h!} 

\newpage       
Da nun, die beiden Koeffizienten in der Zeile cT nicht positiv sind ist der Vorgang abgeschlossen und wir können in der folgenden Abbildung unsere Ergebnisse auslesen.       

\myfig{Simplex_10.png}
       {width=1.0\textwidth}
       {Simplex-Tableau 10}
       {Simplex-Tableau 10}
       {Simplex-Tableau 10}
       {h!} 
       
Wie auch bereits im vorherigen Abschnitt, ist die optimale Lösung $x_1 = 100$ und $x_2 = 250$! Das rot markierte Feld gibt uns hierbei den negativen Wert der Zielfunktion an. Die Ergebnisse stimmen mit denen aus 2.1)a) überein!       

\newpage
\subsection{Optimierung eines Kraftwerkparks}

Gegeben ist ein Wärmesystem, wobei die Nachfrage bei einer gewissen Stunde 125MW beträgt. Zur Deckung der
Nachfrage stehen folgende Kraftwerke zur Verfügung:

\begin{enumerate}
\item Steinkohle mit 30MW
\item Gasturbine mit 60MW
\item Biomasse mit 40MW
\end{enumerate}

Weiter Parameter der Kraftwerke sind:
 
\begin{table}[h]
\begin{center}

\begin{tabular}{|c|c|c|c|c|}
\hline 
Technologie & Var.Wartungskosten[€/MWh$_{therm}$] & $\eta$ & Emissionsfaktor[t$_{CO2}$/MWh$_{prim}$] \\ 
\hline 
Steinkohle & 2.5 & 0.36 & 0.34 \\ 
\hline 
Gasturbine & 1.7 & 0.46 & 0.2  \\ 
\hline 
Biomasse & 2.9 & 0.80 & 0 \\ 
\hline
 
\end{tabular} 
\end{center}
\caption{Parameter}
\label{modell1koeff}
\end{table}


Die Brennstoffkosten von Steinkohle betragen 8[€/MWh$_{prim}$],die von Gas 30[€/MWh$_{prim}$] und die von Biomasse 60 [€/MWh$_{prim}$]. 



\subsubsection{Optimierungsmodell}

\paragraph{i): Formulierung des kostenminimalen Optimierungsmodells}\mbox{}\\

Zunächst wird die Zielfunktion formuliert:


\begin{align}
\min_{x}\ z &= (K_{v,Ko} + \frac{K_{Brenn,Ko}}{\eta_{Ko}}) \cdot L_{Ko} + (K_{v,Gas} + \frac{K_{Bren.,Gas}}{\eta_{Gas}}) \cdot L_{Gas} + (K_{v,Bio} + \frac{K_{Brenn,Bio}}{\eta_{Bio}}) \cdot L_{Bio}
\end{align}

%$c=\begin{bmatrix}
%1 \\ 
%2 \\ 
%3
%\end{bmatrix}$

Somit entsprechen:\\


$c^T=\begin{bmatrix}
(K_{v,Ko} + \frac{K_{Brenn,Ko}}{\eta_{Ko}}) & (K_{v,Gas} + \frac{K_{Brenn,Gas}}{\eta_{Gas}}) & (K_{v,Bio} + \frac{K_{Brenn,Bio}}{\eta_{Bio}})
\end{bmatrix}$= $\begin{bmatrix} 24.722 & 66.917 & 77.9 \end{bmatrix} $

und $x=\begin{bmatrix}
L_{Kohle} \\ 
L_{Gas} \\ 
L_{Bio}
\end{bmatrix}$ 
den Vektoren der allgemeinen Formulierung.\\
Um dieses Minimierungsproblem mittels Tabular-Methode zu lösen, wandeln wir die Zielfunktion mittels
Dualitätssatz:


\begin{equation}
\max_{x}\ f(x) = \min_{x}\ -f(x) 
\end{equation}

in ein Maximierungsproblem um.\\
Zusammen mit den Nebenbediengungen:

\begin{equation}
L_{Kohle} \leq 30MW
\end{equation}

\begin{equation}
L_{Gasturbine} \leq 60MW
\end{equation}

\begin{equation}
L_{Biomasse} \leq 40MW
\end{equation}

\begin{equation}
L_{Kohle} + L_{Gasturbine} + L_{Biomasse} \geq 125MW
\end{equation}

und der Bedingung dass alle Basisvariablen positive Werte annehmen müssen:
\begin{equation}
L_{Kohle},L_{Gasturbine},L_{Biomasse} \geq 0MW
\end{equation}

kann das lineare Optimierungsmodell formuliert werden.

Aus den Nebenbedingungen lässt sich die Matrix $A=\begin{bmatrix}
1 & 0 & 0 \\ 
0 & 1 & 0 \\ 
0 & 0 & 1 \\ 
1 & 1 & 1
\end{bmatrix}$ und $b=\begin{bmatrix}
30 \\ 
60 \\ 
40 \\ 
125
\end{bmatrix}$ direkt ablesen.




\paragraph{ii): Formulierung des emissionsminimalen Optimierungsmodells}\mbox{}\\


Wiederrum zunächst die Formulierung der Zielfunktion: 



\begin{align}
\min_{x}\ z = E_{Kohle} \cdot \frac{L_{Kohle}}{\eta_{Ko}} + E_{Gas} \cdot \frac{L_{Gas}}{\eta_{Gas}}
\end{align}

Da das Biomasse-Kraftwerk einen Emissionsfaktor von 0 hat verschwindet die Abhängigkeit in der Zielfunktion.

Somit entsprechen $c^T=\begin{bmatrix}
\frac{E_{Kohle}}{\eta_{Ko}} & \frac{E_{Gas}}{\eta_{Gas}}
\end{bmatrix}$ =
$\begin{bmatrix}
0.944 & 0.435
\end{bmatrix}$ 
 und $x=\begin{bmatrix}
 L_{Kohle} \\ 
 L_{Gas}
 \end{bmatrix}$

den Vektoren der allgemeinen Formulierung.\\

Um dieses Minimierungsproblem mittels Tabular-Methode zu lösen, wandeln wir die Zielfunktion mittels
Dualitätssatz:

\begin{equation}
\max_{x}\ f(x) = \min_{x}\ -f(x) 
\end{equation}


in ein Maximierungsproblem um.


Zusammen mit den Nebenbediengungen:

\begin{equation}
L_{Kohle} \leq 30MW
\end{equation}

\begin{equation}
L_{Gasturbine} \leq 60MW
\end{equation}

\begin{equation}
L_{Biomasse} \leq 40MW
\end{equation}

\begin{equation}
L_{Kohle} + L_{Gasturbine} + L_{Biomasse} \geq 125MW
\end{equation}

und der Bedingung dass alle Basisvariablen positive Werte annehmen müssen:
\begin{equation}
L_{Kohle},L_{Gasturbine},L_{Biomasse} \geq 0MW
\end{equation}


kann das lineare Optimierungsmodell formuliert werden.

Aus den Nebenbedingungen lässt sich die Matrix $A=\begin{bmatrix}
1 & 0 & 0 \\ 
0 & 1 & 0 \\ 
0 & 0 & 1 \\ 
1 & 1 & 1
\end{bmatrix}$ und $b=\begin{bmatrix}
30 \\ 
60 \\ 
40 \\ 
125
\end{bmatrix}$ direkt ablesen.







%
% \[
%    \vec{c}=\left(\begin{array}{c} (K_{var,Kohle} + \frac{K_{Brennstoff,Kohle}}{\eta_{Kohle}}) \\ (K_{var,Gas} + \frac{K_{Brennstoff,Gas}}{\eta_{Gas}}) \\ (K_{var,Bio} + \frac{K_{Brennstoff,Bio}}{\eta_{Bio}}) \end{array}\right)
%  \]

%% \qquad
%%    D(\varphi)=\left(\begin{array}{rr} \cos\varphi & -\sin\varphi \\
%%      \sin\varphi & \cos\varphi\end{array}\right) .



\subsubsection{Lösung mittels 2-Phasen-Simplexalgorithmus}


Nun wird das lineare Optimierungsproblems mittels Simplex-Methode gelöst.



Hierfür werden die Nebenbedingungen um den Schlupfvektor
$u=\begin{bmatrix}
u_1 \\ 
u_2 \\ 
u_3 \\ 
u_4

\end{bmatrix}$

erweitert

Daraus ergeben sich folgende Nebenbedingungen:


\begin{equation}
L_{Kohle} + u_1 =  30
\end{equation}

\begin{equation}
L_{Gasturbine}+ u_2 = 60
\end{equation}

\begin{equation}
L_{Biomasse} + u_3 = 40
\end{equation}

\begin{equation}
L_{Kohle} + L_{Gasturbine} + L_{Biomasse} - u_4 = 125
\end{equation}


\myfig{ersteslatex.png}
       {width=1.0\textwidth}
       {Start-Simplex-Tableau}
       {Start-Simplex-Tableau}
       {Start-Simplex-Tableau}
       {h!}


Nun sehen wir dass in der Zielfunktionszeile lediglich negative Werte vorkommen. Dies stellt eine nicht zulässige Basislösung dar. Durch Addition einer Hilfsvariablen $a_i$ zu jeder Zeile, in der eine Schlupfvariable subtrahiert wird erhalten wir eine zulässige Basislösung. Die neue Zielfunktion besteht aus der Summe aller $a_i$. 
\newpage

\myfig{zweiteslatex.png}
       {width=1.0\textwidth}
       {Neue Zielfunktion}
       {Neue Zielfunktion}
       {Neue Zielfunktion}
       {h!}
       
       
       
Durch Addition aller Zeilen, die eine Hilfsvariable enthalten, zur Zielfunktionszeile
werden alle Einträge der Hilfsvariablen $a_i$
in der Zielfunktionszeile auf Null
gebracht.In diesem Fall wird die letzte Zeile zur Zielfunktionszeile addiert.
   
\myfig{dritteslatex.png}
       {width=1.0\textwidth}
       {Nullsetzen der $a_i$}
       {Nullsetzen der $a_i$}
       {Nullsetzen der $a_i$}
       {h!}      
       
Als nächstes wird die Hilfszielfunktion mit dem Simplex-Standardverfahren minimiert.

Zur Auswahl der Pivotspalte suchen wir in der Zielfunktionszeile den höchsten Wert, wofür die ersten drei Spalten in Frage
kommen. Die Pivotzeile ist diejenige, in der sich der kleinste Wert der Division von b mit der Pivot-Spalte befindet.    
       
\myfig{vierteslatex.png}
       {width=1.0\textwidth}
       {Auswahl der Pivotspalte-und-zeile}
       {Auswahl der Pivotspalte-und-zeile}
       {Auswahl der Pivotspalte-und-zeile}
       {h!}     
       
Da das Pivotelement bereits normiert ist müssen nur noch die anderen Elemente der Pivotspalte durch Elimination auf null gebracht werden.       
       
      
       
\myfig{fünfteslatex.png}
       {width=1.0\textwidth}
       {Nullsetzen der restlichen Elemente}
       {Nullsetzen der restlichen Elemente}
       {Nullsetzen der restlichen Elemente}
       {h!}          
       
       
Für die beiden anderen Spalten erfolgt die Vorgehensweise analog.
       
       
\myfig{sechsteslatex.png}
       {width=1.0\textwidth}
       {Auswahl der Pivotspalte-und-zeile}
       {Auswahl der Pivotspalte-und-zeile}
       {Auswahl der Pivotspalte-und-zeile}
       {h!}          




       
\myfig{siebteslatex.png}
       {width=1.0\textwidth}
       {Nullsetzen der restlichen Elemente}
       {Nullsetzen der restlichen Elemente}
       {Nullsetzen der restlichen Elemente}
       {h!} 

\newpage
\myfig{achteslatex.png}
       {width=1.0\textwidth}
       {Auswahl der Pivotspalte-und-zeile}
       {Auswahl der Pivotspalte-und-zeile}
       {Auswahl der Pivotspalte-und-zeile}
       {h!} 
       

\newpage
\myfig{neunteslatex.png}
       {width=1.0\textwidth}
       {Nullsetzen der restlichen Elemente}
       {Nullsetzen der restlichen Elemente}
       {Nullsetzen der restlichen Elemente}
       {h!} 









Nachdem die Hilfszielfunktion null erreicht hat, ist die Hilfsvariable $a_1$ keine Basisvariable mehr und wir ersetzen die Hilfszielfunktion durch die originale Zielfunktion.


Schlussendlich wird die angepasste Zielfunktion $c^T$’ berechnet. 



\begin{equation}
c^{T'} = c^T + 24.722 \cdot Zeile1 + 66.917 \cdot Zeile2 + 77.9 \cdot Zeile4
\end{equation}

% = c^T + 22.722 \cdot Zeile1 + 66.917 \cdot Zeile2 + 77.9 \cdot Zeile4

\myfig{zehnteslatex.png}
       {width=1.0\textwidth}
       {Berechnung der angepassten Zielfunktion}
       {Berechnung der angepassten Zielfunktion}
       {Berechnung der angepassten Zielfunktion}
       {h!} 


Aus dieser Tabelle sind nun die einzelnen Leistungen der kostenminimalen Optimierung entnehmbar. Das Steinkohlekraftwerk und das Gasturbinenkraftwerk wird vollastig mit jeweils \textbf{30MW} und \textbf{60MW} betrieben. Das Biomassekraftwerk mit \textbf{35MW}.
Die Gesamtkosten des Kraftwerkparks ergeben sich zu \textbf{7483.18 EUR/h} für eine Erzeugung von \textbf{125MW.}











\newpage

\newpage
\subsection{Schadstoffreduzierung eines Kraftwerks}
\subsubsection{mathematisches Modell}
Unser mathematisches Modell, welches die Kostenreduziert und gleichzeitig die gewünschten Emissionsreduktionen einhält, sieht wie folgt aus:

\begin{align}
\min_{\textbf{x}}\ z = \textbf{c}^T \cdot \textbf{x}
\end{align}

Wobei die Variablen wie folgt definiert sind:

\begin{center}
\parbox{13cm}{
$z$ ... die Zielfunktion (die Gesamtkosten in Millionen €)

$\textbf{c}$ ... Gewichtungskonstanten (Kosten pro Maßnahmentyp in Millionen €)
\begin{center}
\parbox{9cm}{
$\textbf{c}^T = \begin{matrix}(8&10&7&6&11&9)\end{matrix}$
}
\end{center}
$\textbf{x}$ ... Entscheidungsvariablen-Vektor (Werte zwischen 0 und 1)


\begin{center}
\parbox{9cm}{
$x_1$ ... Schornstein Typ 1\\
$x_2$ ... Schornstein Typ 2\\
$x_3$ ... Filter Typ 1\\
$x_4$ ... Filter Typ 2\\
$x_5$ ... Brennstoff Typ 1\\
$x_6$ ... Brennstoff Typ 2\\
}
\end{center}
}
\end{center}

Die Nebenbedingungen sind die Emissionsreduktionsbedingungen (in Tonnen):
\begin{align}
\label{Nebenbedinung_StaubRuss}12 \cdot x_1 + 9 \cdot x_2  + 25 \cdot x_3 + 20 \cdot x_4 + 17 \cdot x_5  + 13 \cdot x_6 \geq 60\\
\label{Nebenbedinung_Schwefeloxide}35 \cdot x_1 + 42 \cdot x_2  + 18 \cdot x_3 + 31 \cdot x_4 + 56 \cdot x_5  + 49 \cdot x_6 \geq 150\\ 
\label{Nebenbedinung_Kohlenwasserstoffe}37 \cdot x_1 + 53 \cdot x_2  + 28 \cdot x_3 + 24 \cdot x_4 + 29 \cdot x_5  + 20 \cdot x_6 \geq 125
\end{align}

Wobei sich die Nebenbedingungsgleichung \ref{Nebenbedinung_StaubRuss} auf die Staub und Ruß-, \ref{Nebenbedinung_Schwefeloxide} auf die Schwefeloxid- und \ref{Nebenbedinung_Schwefeloxide} auf die Kohlenwasserstoffe-Emissionsreduktion bezieht.
\newpage
\subsubsection{Lösung mittels MATLAB}
Das Optimierungsproblem wurde mittels MATLAB mit Hilfe von yalmip und als Löser "Gurobi" gelöst.

Anbei findet man die kostenoptimale Umsetzung der verschiedenen Maßnahmen.

\begin{center}
\parbox{9cm}{
$x_1$ = 100\%\\
$x_2$ = 62.27\%\\
$x_3$ = 34.35\%\\
$x_4$ = 100.00\%\\
$x_5$ = 4.76\%\\
$x_6$ = 100.00\%
}
\end{center}

Die daraus resultierenden Optimierungskosten betragen 32.155 Millionen €. 

\subsubsection{Lösung des MILP}
Da die Maßnahmen nicht zu einem beliebigen Grad umgesetzt werden können, soll nun ein MILP (Mixed Integer Linear Problem) die Realität besser abbilden.

Die folgenden Werte wurden wieder mittels MATLAB berechnet.

\begin{center}
\parbox{9cm}{
$x_1$ = 100\%\\
$x_2$ = 100\%\\
$x_3$ = 100\%\\
$x_4$ = 0\%\\
$x_5$ = 100\%\\
$x_6$ = 0\%
}
\end{center}

Die daraus resultierenden Optimierungskosten betragen 36 Millionen €. 

\subsubsection{Erweiterung des MILP}
Die zusätzliche entweder/oder-Bedingung wurde in MATLAB berücksichtigt und zu 0 berechnet. 

Wie aus der Tabellenangabe erkennbar, reichen 3 der ursprünglichen 6 Maßnahmen nicht aus um die Emissionsreduktionsbedingungen zu erfüllen.
\newpage
\appendix


\newpage
%auskommentieren wenn notwendig:

\listoffigures   
\listoftables
%\printbibliography

\end{document}
